\documentclass{article}
\usepackage[utf8]{inputenc}
\usepackage{listings}
\lstset{
	language=Octave,
	frame=single,
	xleftmargin=.1\textwidth, xrightmargin=.1\textwidth
}
\usepackage[T2A]{fontenc}
\usepackage[utf8]{inputenc}
\usepackage[russian]{babel}
\usepackage[left=2cm,right=2cm,top=2cm,bottom=2.1cm,bindingoffset=0cm]{geometry}

\title{Домашнее задание 1}
\author{Кокорин Илья, M3439}
\date{Вариант № 64}

\begin{document}
	\maketitle
	\section{Начальные условия}
	Проверочная матрица
	$
	H = \left(
	\begin{array}{cccccccccc}
	0 & 0 & 0 & 0 & 1 & 0 & 0 & 1 & 1 & 1\\
	1 & 0 & 1 & 1 & 1 & 0 & 1 & 0 & 1 & 1\\
	0 & 1 & 0 & 1 & 0 & 0 & 1 & 1 & 0 & 1\\
	1 & 0 & 0 & 1 & 0 & 1 & 0 & 0 & 1 & 0
	\end{array}
	\right)
	$
	
	\section{Определить скорость кода}
	
	$r = 4$ (количество строк проверночной матрицы)
	
	$n = 10$ (количество столбцов проверочной и порождающей матриц)
	
	$r = n - k \Rightarrow k = n - r = 10 - 4 = 6$ (количество строк порождающей матрицы)
	
	Скорость кода $R = \frac{k}{n} = \frac{6}{10} = 0.6$ бит/cимвол
	
	\section{Определить минимальное расстояние кода}
	
	По теореме 2.4 минимальное расстояние линейного $(n, k)$-кода равно $d$ в том и только в том случае, когда любые $d - 1$ столбцов проверочной матрицы линейно независимы и существует набор из $d$ линейно зависимых столбцов.
	
	Любые два столбца проверочной матрицы $H$ линейно независимы, так как все столбцы матрицы $H$ различны (в $F_2$ это эквивалентно линейной независимости двух векторов).
	
	Заметим также, что первый, второй и четвертый столбцы матрицы $H$ линейно зависимы
	
	$
	\begin{array}{c}
	0\\
	1\\
	0\\
	1
	\end{array} + \begin{array}{c}
	0\\
	0\\
	1\\
	0
	\end{array} + \begin{array}{c}
	0\\
	1\\
	1\\
	1
	\end{array} = \begin{array}{c}
	0\\
	0\\
	0\\
	0
	\end{array}
	$
	
	Поэтому d = 3
	
	\section{Построить порождающую матрицу}
	Приведём матрицу $H$ тривиальными преобразованиями (сменой строк и заменой строки её линейной комбинацией с другими строками) к виду $(P^T I_r)$
	
	Исходная матрица:
	
	$
	\begin{array}{cccccccccc}
	0 & 0 & 0 & 0 & 1 & 0 & 0 & 1 & 1 & 1\\
	1 & 0 & 1 & 1 & 1 & 0 & 1 & 0 & 1 & 1\\
	0 & 1 & 0 & 1 & 0 & 0 & 1 & 1 & 0 & 1\\
	1 & 0 & 0 & 1 & 0 & 1 & 0 & 0 & 1 & 0
	\end{array}
	$
	
	Меняем 3 и 4 строки местами
	
	$
	\begin{array}{cccccccccc}
	0 & 0 & 0 & 0 & 1 & 0 & 0 & 1 & 1 & 1\\
	1 & 0 & 1 & 1 & 1 & 0 & 1 & 0 & 1 & 1\\
	1 & 0 & 0 & 1 & 0 & 1 & 0 & 0 & 1 & 0 \\
	0 & 1 & 0 & 1 & 0 & 0 & 1 & 1 & 0 & 1
	\end{array}
	$
	
	Заменяем вторую строку на линейную комбинацию второй и четвёртой
	
	$
	\begin{array}{cccccccccc}
	0 & 0 & 0 & 0 & 1 & 0 & 0 & 1 & 1 & 1\\
	1 & 1 &1 &0 &1 &0 &0 &1 &1 &0\\
	1 & 0 & 0 & 1 & 0 & 1 & 0 & 0 & 1 & 0 \\
	0 & 1 & 0 & 1 & 0 & 0 & 1 & 1 & 0 & 1
	\end{array}
	$
	
	Заменяем первую строку на линейную комбинацию первой и четвёртой
	
	$
	\begin{array}{cccccccccc}
	0 &1& 0& 1& 1& 0& 1& 0& 1& 0\\
	1 & 1 &1 &0 &1 &0 &0 &1 &1 &0\\
	1 & 0 & 0 & 1 & 0 & 1 & 0 & 0 & 1 & 0 \\
	0 & 1 & 0 & 1 & 0 & 0 & 1 & 1 & 0 & 1
	\end{array}
	$
	
	Заменим вторую строку на линейную комбинацию второй и третьей.
	
	$
	\begin{array}{cccccccccc}
	0 &1& 0& 1& 1& 0& 1& 0& 1& 0\\
	0 & 1 & 1 & 1 & 1 & 1 & 0 & 1 & 0 & 0\\
	1 & 0 & 0 & 1 & 0 & 1 & 0 & 0 & 1 & 0 \\
	0 & 1 & 0 & 1 & 0 & 0 & 1 & 1 & 0 & 1
	\end{array}
	$
	
	Заменим первую строку на линейную комбинацию первой и третьей
	
	$
	\begin{array}{cccccccccc}
	1 & 1 & 0& 0 & 1 & 1 & 1 & 0 & 0 & 0\\
	0 & 1 & 1 & 1 & 1 & 1 & 0 & 1 & 0 & 0\\
	1 & 0 & 0 & 1 & 0 & 1 & 0 & 0 & 1 & 0 \\
	0 & 1 & 0 & 1 & 0 & 0 & 1 & 1 & 0 & 1
	\end{array}
	$
	
	Заменим четвёртую строку на линейную комбинацию первой и четвёртой
	
	$
	\begin{array}{cccccccccc}
	1 & 1 & 0& 0 & 1 & 1 & 1 & 0 & 0 & 0\\
	0 & 1 & 1 & 1 & 1 & 1 & 0 & 1 & 0 & 0\\
	1 & 0 & 0 & 1 & 0 & 1 & 0 & 0 & 1 & 0 \\
	1 &0 &0 &1 &1 &1 &0 &1 &0 &1
	\end{array}
	$
	
	Заменим четвёртую строку на линейную комбинацию второй и четвёртой
	
	$
	\begin{array}{cccccccccc}
	1 & 1 & 0& 0 & 1 & 1 & 1 & 0 & 0 & 0\\
	0 & 1 & 1 & 1 & 1 & 1 & 0 & 1 & 0 & 0\\
	1 & 0 & 0 & 1 & 0 & 1 & 0 & 0 & 1 & 0 \\
	1 & 1 & 1 & 0 & 0 & 0 & 0 & 0 & 0 & 1
	\end{array}
	$
	
	Тогда
	
	$
	P^T = \left(
	\begin{array}{cccccc}
	1 & 1 & 0& 0 & 1 & 1\\
	0 & 1 & 1 & 1 & 1 & 1 \\
	1 & 0 & 0 & 1 & 0 & 1\\
	1 & 1 & 1 & 0 & 0 & 0
	\end{array}
	\right)
	$
	
	Тогда $
	P = \left(
	\begin{array}{cccccc}
	1& 0& 1& 1\\
	1& 1& 0& 1\\
	0 &1 &0 &1\\ 
	0 &1 &1 &0\\
	1 &1 &0 &0\\
	1 &1 &1 &0
	\end{array}
	\right)
	$
	
	Тогда $
	G = \left(
	\begin{array}{cccccccccc}
	1& 0& 0& 0& 0& 0   & 1& 0& 1& 1  \\
	0& 1& 0& 0& 0& 0   & 1& 1& 0& 1  \\
	0 &0 &1 &0 &0 &0   & 0 &1 &0 &1 \\ 
	0 &0 &0 &1 &0 &0   & 0 &1 &1 &0 \\
	0 &0 &0 &0 &1 &0   & 1 &1 &0 &0  \\
	 0 &0 &0 &0 &0 &1  & 1 &1 &1 &0 
	\end{array}
	\right)
	$
	
	Проведём проверку:
	
	\begin{lstlisting}[language=python]
g = [
	[1, 0, 0, 0, 0, 0, 1, 0, 1, 1],
	[0, 1, 0, 0, 0, 0, 1, 1, 0, 1],
	[0, 0, 1, 0, 0, 0, 0, 1, 0, 1],
	[0, 0, 0, 1, 0, 0, 0, 1, 1, 0],
	[0, 0, 0, 0, 1, 0, 1, 1, 0, 0],
	[0, 0, 0, 0, 0, 1, 1, 1, 1, 0]
]


def transpose(matrix):
	n = len(matrix)
	m = len(matrix[0])
	transposed = [[None for _ in range(n)] for _ in range(m)]

	for i in range(n):
		for j in range(m):
			transposed[j][i] = matrix[i][j]
	return transposed


h = [
	[0, 0, 0, 0, 1, 0, 0, 1, 1, 1],
	[1, 0, 1, 1, 1, 0, 1, 0, 1, 1],
	[0, 1, 0, 1, 0, 0, 1, 1, 0, 1],
	[1, 0, 0, 1, 0, 1, 0, 0, 1, 0]
]

h_t = transpose(h)

n = len(g)
m = len(g[0])
assert m == len(h_t)
k = len(h_t[0])


result = [[0 for _ in range(k)] for _ in range(n)]

for i in range(n):
	for j in range(k):
		for z in range(m):
			result[i][j] += g[i][z] * h_t[z][j]
			result[i][j] %= 2

print(result)
# [[0, 0, 0, 0], [0, 0, 0, 0], [0, 0, 0, 0],
 [0, 0, 0, 0], [0, 0, 0, 0], [0, 0, 0, 0]]
	\end{lstlisting}
	
	Проверка показывает, что $G \cdot H^T = 0$
	
	\section{Закодировать произвольное сообщение}
	
	Пусть 
	
	$
	m = \left(
	\begin{array}{cccccc}
	1 & 0 & 0 & 0 & 1 & 0
	\end{array}
	\right)
	$
	
	Тогда $c = m \cdot G = \begin{array}{cccccc}
	\left(
	\begin{array}{cccccc}
	1 & 0 & 0 & 0 & 1 & 0
	\end{array}
	\right)
	\end{array} \cdot  \left(
	\begin{array}{cccccccccc}
	1& 0& 0& 0& 0& 0   & 1& 0& 1& 1  \\
	0& 1& 0& 0& 0& 0   & 1& 1& 0& 1  \\
	0 &0 &1 &0 &0 &0   & 0 &1 &0 &1 \\ 
	0 &0 &0 &1 &0 &0   & 0 &1 &1 &0 \\
	0 &0 &0 &0 &1 &0   & 1 &1 &0 &0  \\
	0 &0 &0 &0 &0 &1  & 1 &1 &1 &0 
	\end{array}
	\right)$
	
	Тогда $
	c = \left(
	\begin{array}{cccccccccc}
	1 & 0 &0 &0 &1 &0 &0 &1 &1 &1
	\end{array}
	\right)
	$
\end{document}