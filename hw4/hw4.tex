\documentclass{article}
\usepackage[utf8]{inputenc}
\usepackage{listings}
\lstset{
	language=Octave,
	frame=single,
	xleftmargin=.1\textwidth, xrightmargin=.1\textwidth
}
\usepackage{graphicx}
\usepackage{mathtools, nccmath}
\usepackage[T2A]{fontenc}
\usepackage[utf8]{inputenc}
\usepackage[russian]{babel}
\usepackage{amsmath}
\usepackage[left=2cm,right=2cm,top=2cm,bottom=2.1cm,bindingoffset=0cm]{geometry}
\usepackage{amsfonts}

\DeclarePairedDelimiter{\nint}\lfloor\rfloor

\title{Домашнее задание 4}
\author{Кокорин Илья, M3439}
\date{Вариант № 64}

\begin{document}
	\maketitle
	\section{Задание 3}
	Проверочная матрица
	$
	H = \left(
	\begin{array}{cccccccccc}
	0 & 0 & 0 & 0 & 1 & 0 & 0 & 1 & 1 & 1\\
	1 & 0 & 1 & 1 & 1 & 0 & 1 & 0 & 1 & 1\\
	0 & 1 & 0 & 1 & 0 & 0 & 1 & 1 & 0 & 1\\
	1 & 0 & 0 & 1 & 0 & 1 & 0 & 0 & 1 & 0
	\end{array}
	\right)
	$
	
	Порождающая матрица $
	G = \left(
	\begin{array}{cccccccccc}
	1& 0& 0& 0& 0& 0   & 1& 0& 1& 1  \\
	0& 1& 0& 0& 0& 0   & 1& 1& 0& 1  \\
	0 &0 &1 &0 &0 &0   & 0 &1 &0 &1 \\ 
	0 &0 &0 &1 &0 &0   & 0 &1 &1 &0 \\
	0 &0 &0 &0 &1 &0   & 1 &1 &0 &0  \\
	0 &0 &0 &0 &0 &1  & 1 &1 &1 &0 
	\end{array}
	\right)
	$
	
	$k = 6, n = 10, r = 4, d = 3$
	
	Определим диапазон оптимальных значений d для данного кода.
	
	\subsection{Граница Хэмминга}
	
	Граница Хемминга имеет вид $M \leq \frac{q^n}{\sum\limits_{i = 0}^t \binom{n}{i} \cdot (q - 1)^i}$, в случае двоичного кода, у которого $M = 2^k, q = 2$, принимает вид $2^k \leq \frac{2^n}{\sum\limits_{i = 0}^t \binom{n}{i} \cdot (2 - 1)^i} \Rightarrow 2^k \leq \frac{2^n}{\sum\limits_{i = 0}^t \binom{n}{i}} \Rightarrow \sum\limits_{i = 0}^t \binom{n}{i} \leq 2^{n - k}$
	
	При заданных $n, k = 10, 6$ принимает вид: $\sum\limits_{i = 0}^t \binom{10}{i} \leq 16$
	
	Побрерём перебором такое максимальное $d$, что при $t = \nint{\frac{d - 1}{2}}$ формула выполняется.
	
	\begin{lstlisting}[language=python]
import math


def c(n, k):
	return math.factorial(n) // 
		(math.factorial(k) * math.factorial(n - k))


def check_hamming(n, k, d):
	t = (d - 1) // 2
	total_sum = 0
	for i in range(t + 1):
		total_sum += c(n, i)
	return total_sum <= 2 ** (n - k)


cur_d = 1
n = 10
k = 6

while True:
	if not check_hamming(n, k, cur_d):
		print(cur_d - 1)  # 4
		break
	else:
		cur_d += 1
	\end{lstlisting}
	
	По теореме Хемминга, верхней границей для d является 4, то есть ни для какого $d' > 4$ не существует двоичный линейный $(n, k)$-код с минимальным расстоянием $d'$.

	\subsection{Граница Варшамова-Гилберта}
	
	Граница Варшамова-Гилберта имеет вид $q^{n - k} > \sum\limits_{i = 0}^{d - 2} \binom{n - 1}{i} \cdot (q - 1)^i$
	
	Что в нашем случае, превращается в $2^{10 - 6} > \sum\limits_{i = 0}^{d - 2} \binom{10 - 1}{i} \cdot (2 - 1)^i \Rightarrow  16 > \sum\limits_{i = 0}^{d - 2} \binom{9}{i} \Rightarrow \sum\limits_{i = 0}^{d - 2} \binom{9}{i} < 16$
	
	Найдём с помощью перебора такое максимальное $d$, для которого эта формула выполяется. Тогда, по теореме Варшамова-Гилберта, для такого $d$ будет существовать двоичный линейный $(n, k)$-код с минимальным расстоянием d.
	
	Приведём программу для поиска такой границы.
	
	\begin{lstlisting}[language=python]
import math


def c(n, k):
	return math.factorial(n) // 
		(math.factorial(k) * math.factorial(n - k))	
	
	
def check_hilbert(n, k, d):
	total_sum = 0
	for i in range(d - 1):
		total_sum += c(n - 1, i)
	return total_sum < 2 ** (n - k)


n = 10
k = 6
cur_d = 1
while True:
	if not check_hilbert(n, k, cur_d):
		print(cur_d - 1)  # 3
		break
	else:
		cur_d += 1
	\end{lstlisting}
	
	То есть для $d = 3$ существует двоичный линейный $(n, k)$-код с мнимальным расстоянием $d$.
	
	\subsection{Проверка кода на оптимальность}
	Тогда минимальное расстояние оптимального кода следует искать во множестве $\{d: 3 \leq d \leq 4\}$
	
	Минимальное расстояние исследуемого кода попадает в данный диапазон.
	
	Как следует из материалов сайта $http://www.codetables.de/$ и таблицы $3.2$, лучший код для $n = 10, k = 6$ имеет $d = 3$, то есть приведённый код является оптимальным.
	
	\section{Задача 3.3}
	
	Зафиксируем $n, d = 31, 7$
	
	\subsection{Граница Хемминга}
	
	 $t = \nint{\frac{d - 1}{2}} = 3$
	 
	 Граница Хемминга имеет вид $M \leq \frac{q^n}{\sum\limits_{i = 0}^t \binom{n}{i} \cdot (q - 1)^i}$, в случае двоичного кода, у которого $M = 2^k, q = 2$, принимает вид $2^k \leq \frac{2^n}{\sum\limits_{i = 0}^t \binom{n}{i} \cdot (2 - 1)^i} \Rightarrow 2^k \leq \frac{2^n}{\sum\limits_{i = 0}^t \binom{n}{i}} = \frac{2^{31}}{\binom{31}{0} + \binom{31}{1}  + \binom{31}{2}  + \binom{31}{3} } = \frac{16777216}{39}$
	 
	 Тогда $k = \nint{log_2(\frac{16777216}{39})} = 18$
	 
	 $k = 18$ является верхней границей для $k$, то есть ни для какого $k' > 18$ не существует двоичного линейного $(n, k)$-кода с заданными параметрами.
	 
	 \subsection{Граница Варшамова-Гилберта}
	 
	 Граница Варшамова-Гилберта имеет вид $q^{n - k} > \sum\limits_{i = 0}^{d - 2} \binom{n - 1}{i} \cdot (q - 1)^i$
	 
	 Что в нашем случае, превращается в $2^{31 - k} > \sum\limits_{i = 0}^{7 - 2} \binom{31 - 1}{i} \cdot (2 - 1)^i \Rightarrow  2^{31 - k} > \sum\limits_{i = 0}^{5} \binom{30}{i} \Rightarrow 2^{31 - k} > 174437 \Rightarrow 31 - k > log_2(174437) \Rightarrow 31 - k \geq 18 \Rightarrow k \leq 13$
	 
	 Тогда при $k \leq 13$ точно существует код с вышеуказанными параметрами. Значит, такой код гарантированно существует для $k = 13$. Тогда $k = 13$ является нижней границей для оптимального $k$.
	 
	 \subsection{Определение оптимального кода}
	 
	 Так как код с вышеуказанными параметрами гарантированно существует при $k \leq 13$ (а значит,  и при $k = 13$), и точно не существует при $k > 18$, то k для оптимального кода нужно искать во множестве $\{k: 13 \leq k \leq 18\}$
	 
	 Из таблицы 3.2 следует, что при $n = 31$ минимальное расстояние $d = 7$ достигается при $k = 17$, $17 \in \{k: 13 \leq k \leq 18\}$
	 
	 \section{Задача 3.4}
	 
	 Выберем  $n = 25, k = 17$
	 
	 \subsection{Граница Хемминга}
	 
	 Граница Хемминга имеет вид $M \leq \frac{q^n}{\sum\limits_{i = 0}^t \binom{n}{i} \cdot (q - 1)^i}$, в случае двоичного кода, у которого $M = 2^k, q = 2$, принимает вид $2^k \leq \frac{2^n}{\sum\limits_{i = 0}^t \binom{n}{i} \cdot (2 - 1)^i} \Rightarrow 2^k \leq \frac{2^n}{\sum\limits_{i = 0}^t \binom{n}{i}} \Rightarrow \sum\limits_{i = 0}^t \binom{n}{i} \leq 2^{n - k}$
	 
	 Побрерём перебором такое максимальное $d$, что при $t = \nint{\frac{d - 1}{2}}$ формула выполняется.
	 
	 \begin{lstlisting}[language=python]
import math


def c(n, k):
	return math.factorial(n) // 
		(math.factorial(k) * math.factorial(n - k))


def check_hamming(n, k, d):
	t = (d - 1) // 2
	total_sum = 0
	for i in range(t + 1):
		total_sum += c(n, i)
	return total_sum <= 2 ** (n - k)


cur_d = 1
n = 25
k = 17

while True:
	if not check_hamming(n, k, cur_d):
		print(cur_d - 1)  # 4
		break
	else:
		cur_d += 1
	 \end{lstlisting}
	 
	 По теореме Хемминга, верхней границей для d является 4, то есть ни для какого $d' > 4$ не существует двоичный линейный $(n, k)$-код с минимальным расстоянием $d'$.
	 
	 \subsection{Граница Варшамова-Гилберта}
	 
	 Граница Варшамова-Гилберта имеет вид $q^{n - k} > \sum\limits_{i = 0}^{d - 2} \binom{n - 1}{i} \cdot (q - 1)^i$
	 
	 Что в нашем случае, превращается в $2^{25 - 17} > \sum\limits_{i = 0}^{d - 2} \binom{25 - 1}{i} \cdot (2 - 1)^i \Rightarrow  256 > \sum\limits_{i = 0}^{d - 2} \binom{24}{i} \Rightarrow \sum\limits_{i = 0}^{d - 2} \binom{24}{i} < 256$
	 
	 Найдём с помощью перебора такое максимальное $d$, для которого эта формула выполяется. Тогда, по теореме Варшамова-Гилберта, для такого $d$ будет существовать двоичный линейный $(n, k)$-код с минимальным расстоянием d.
	 
	 Приведём программу для поиска такой границы.
	 
	 \begin{lstlisting}[language=python]
import math


def c(n, k):
 return math.factorial(n) // 
	 	(math.factorial(k) * math.factorial(n - k))	


def check_hilbert(n, k, d):
	total_sum = 0
	for i in range(d - 1):
		total_sum += c(n - 1, i)
	return total_sum < 2 ** (n - k)

n = 25
k = 17
cur_d = 1
while True:
	if not check_hilbert(n, k, cur_d):
		print(cur_d - 1)  # 3
		break
	else:
		cur_d += 1
	 \end{lstlisting}
	 
	 То есть для $d = 3$ существует двоичный линейный $(n, k)$-код с мнимальным расстоянием $d$.
	 
	 \subsection{Проверка кода на оптимальность}
	 Тогда минимальное расстояние оптимального кода следует искать во множестве $\{d: 3 \leq d \leq 4\}$

	 Как следует из таблицы $3.2$, лучший код для $n = 25, k = 17$ имеет $d = 4$, $4 \in \{d: 3 \leq d \leq 4\}$
	 
\end{document}