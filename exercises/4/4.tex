\documentclass{article}
\usepackage[utf8]{inputenc}
\usepackage{listings}
\lstset{
	language=Octave,
	frame=single,
	xleftmargin=.1\textwidth, xrightmargin=.1\textwidth
}
\usepackage{graphicx}
\usepackage{mathtools, nccmath}
\usepackage[T2A]{fontenc}
\usepackage[utf8]{inputenc}
\usepackage[russian]{babel}
\usepackage{amsmath}
\usepackage[left=2cm,right=2cm,top=2cm,bottom=2.1cm,bindingoffset=0cm]{geometry}
\usepackage{amsfonts}

\graphicspath{{/pic}}
\DeclarePairedDelimiter{\nint}\lfloor\rfloor
\DeclarePairedDelimiter{\hint}\lceil\rceil

\title{Упражнения из учебника}
\author{Кокорин Илья, M3439}
\date{Вариант № 64}

\begin{document}
	\maketitle
	\section{Задача 3.7}
	
	Граница Грайсмера имеет вид $N(k, d) \geq \sum\limits_{i = 0}^{k - 1} \hint{\frac{d}{2^i}}$
	
	Найдём в таблице 3.2 коды с $d = 2$, и для каждого $k$ возьмём минимальное возможное $n$, при котором достигается $d = 8$.
	
	Для таких кодов подсчитаем $N(k, d)$ по формуле Грайсмера, и сравним с числом из таблицы 3.2.
	
	 Код для генерации пграфиков приложен к заданию.
	 
	 Получается следующий график:
	 
	 \includegraphics[scale=1]{pic/ns.png}
	 
	Как видно из графика, при малых $k$, коды лежат на границе Грайсмера, но потом реально достижимая оптимальная длина кода начинает отдаляться от теоретически вычисленной $N(k, d)$ - минимально допустимой длины кода с заданными парамерами, и при больших $k$ теоретически минимальная длина кода (и,  следовательно,скорость кода) не достигнуты на практике. 
\end{document}